\wde{Semi-Decidable} A problem $(D,Q)$ is \emph{semi-decidable} if there is a TM/RM that returns ``yes'' for any $d \in Q$, but may return ``no'' or loop forever when $d \notin Q$.
\wde{Co-Semi-Decidable} A problem $(D, Q)$ is \emph{co-semi-decidable} if there is a TM/RM that returns ``no'' for any $d \notin Q$, but may return ``yes'' or loop forever when $d \in Q$.
\wt{S.D. \& C.S.D. = D} Any problem that is both semi-decidable and co-semi-decidable is decidable.
\wt{Semi-Decidable Complement} If a problem $P$ is semi-decidable, and vice versa.
\wde{Enumerable} A set $S$ is enumerable if there is a bijection between $S$ and $\mathbb{N}$.
\wde{Computably Enumerable} A set $S$ is computably enumerable if the enumeration function $f: \mathbb{N} \to S$ is computable.
\wa{Interleaving} Given $n \in \mathbb{N}$, we can check whether $n = \enc{M}$ for some machine $M$. Therefore, we can enumerate all machines $\enc{M_0},\enc{M_1},\ldots$.
\begin{algorithmic}
    \State $ms := \langle\rangle$; $i := 0$
    \While{true} 
        \State $ms := ms \mathrel{++} \langle\enc{M_i}\rangle$
        \For{$\enc{M} \in ms$}
            \State run $M$ for one step and update $ms$.
            \If{$M$ has halted}
                \State \textbf{output} $\enc{M}$; delete $M$ from $ms$
            \EndIf
        \EndFor
        \State $i := i+1$
    \EndWhile
\end{algorithmic}
\wt{S.D. is C.E.} A problem $P$ is semi-decidable if and only if $P$ is computably enumerable.
\wt{Reduce for C.E.} To prove that a problem $P_2$ is not computably enumerable, show that there is a \emph{mapping} reduction from a known not-computably enumerable problem $P_1$ to $P_2$.
